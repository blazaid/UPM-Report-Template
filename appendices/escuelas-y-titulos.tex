\chapter{Escuelas y títulos}
\label{ch:escuelas-y-titulos}

A continuación se describen todas las opciones de grados y títulos disponibles en la memoria.

\section{Escuelas}

Las escuelas disponibles se describen en el cuadro~\ref{tbl:schools}.

\begin{table}[h]
    \centering
    \begin{tabularx}{\textwidth}{@{}lX@{}}
        \toprule
        \textbf{Clave}  & \textbf{Valor} \\
        \midrule
        %\texttt{etsii}  & E.T.S. de Ingenieros Industriales \\
        \texttt{etsidi} & E.T.S. de Ingeniería y Diseño Industrial \\
        \texttt{etsisi} & E.T.S. de Ingeniería de Sistemas Informáticos \\
        \bottomrule
    \end{tabularx}
    \caption{\label{tbl:schools} Relación entre el código de la plantilla y la escuela a la que se refiere}
\end{table}

De momento no están todas, así que si te apetece añadir la tuya puedes, o bien contactar con los autores, o bien modificarlo (mira el apéndice~\ref{ch:ampliar}) y también contactar con los autores, así lo podemos hacer público con la mayor cantidad de usuarios posible.

\section{Titulaciones}

Cada una de las escuelas poseen ciertas titulaciones que se han de añadir a la configuración.

\subsection{E.T.S. de Ingeniería y Diseño Industrial}

La ETSIDI tiene configurados como colores principal y de link el RGB $(223,30,64)$. El logo es el mostrado la figura~\ref{fig:logo-etsidi}.

\begin{figure}[h]
    \centering
    \includegraphics[width=10em]{upm-report/logos/logo-etsidi}
    \caption{\label{fig:logo-etsidi}Logo de la ETSIDI utilizado en la cubierta trasera de la memoria}
\end{figure}

Las titulaciones que existen la última vez que se actualizó este documento son las siguientes:

\begin{itemize}
    \item \texttt{56IE}: Grado en Ingeniería Eléctrica.
    \item \texttt{56IA}: Grado en Ingeniería Electrónica Industrial y Automática.
    \item \texttt{56IM}: Grado en Ingeniería Mecánica.
    \item \texttt{56IQ}: Grado en Grado en Ingeniería Química.
    \item \texttt{56DD}: Grado en Ingeniería en Diseño Industrial y Desarrollo de Producto.
\end{itemize}

\subsection{E.T.S. de Ingeniería de Sistemas Informáticos}

La ETSISI tiene configurados como colores principal y de link el RGB $(0,177,230)$. El logo es el mostrado la figura~\ref{fig:logo-etsisi}.

\begin{figure}[h]
    \centering
    \includegraphics[width=10em]{upm-report/logos/logo-etsisi}
    \caption{\label{fig:logo-etsisi}Logo de la ETSISI utilizado en la cubierta trasera de la memoria}
\end{figure}

Las titulaciones que existen la última vez que se actualizó este documento son las siguientes:

\begin{itemize}
    \item \texttt{61CD}: Grado en Ciencia de Datos e Inteligencia Artificial.
    \item \texttt{61CI}: Grado en Ingeniería de Computadores.
    \item \texttt{61IW}: Grado en Ingeniería del Software.
    \item \texttt{61SI}: Grado en Sistemas de Información.
    \item \texttt{61TI}: Grado en Tecnologías para la Sociedad de la Información.
\end{itemize}
