\chapter{Introducción}
\label{ch:introduccion}

La introducción a un \gls{pfg}, \gls{pfm} o \gls{td} el el punto de entrada a todo el trabajo realizado y es considerada la más importante tras el abstract, que es quien resume el trabajo entero. En ella habría que dejar claro qué es el trabajo que se ha realizado, por qué es importante y qué es lo que aporta como resultados.

La introducción generará expectativas, y por tanto hay que intentar venderla bien. Un gancho típico en los trabajos suele ser el de aportar un dato relevante o controvertido para discutir sobre él o plantear una pregunta relevante para el contexto en el que se está trabajando.

Dentro del capítulo, tras introducir el trabajo realizado de forma genérica, se suelen incluir las siguientes secciones para establecer bien el alcance y las limitaciones del mismo: motivación, objetivos, suposiciones/limitaciones y, a veces, estructura de la memoria.

Ni que decir tiene que esta estructura planteada, tanto del capítulo como de la memoria en si es únicamente un ejemplo o propuesta. Cada proyecto es único y a veces es más cómodo escribirlo de otro modo.

\section{Objetivos}

El objetivo de un \gls{pfg}, \gls{pfm} y  \gls{td} es una de las piezas clave a plantear, y a su vez una de las más complicadas. Se considera \textbf{la finalidad} del proyecto en cuestión a realizar y suele encajar dentro de una de las siguientes categorías:

\begin{itemize}
    \item \textbf{Contraste} o validación de una hipótesis. Este es típico de \glspl{td}, aunque algunos \glspl{pfm} y (muy raramente) \glspl{pfg} pueden caer dentro de esta categoría.
    \item \textbf{Desarrollo} o diseño de algo (e.g.~Software, hardware, sistema, edificio). Suele ser el más común en la rama de la ingeniería, tanto \glspl{pfm} como \glspl{pfg}.
    \item \textbf{Estudio} de un tema que deduce o descubre nuevo conocimiento. Éste suele ser más común en las ramas de las ciencias puras y humanidades, tanto \glspl{pfm} como \glspl{pfg}.
\end{itemize}

Decimos que es una pieza clave porque sirve como primer indicador de la consecución del proyecto. Si nos planteamos un objetivo, en las conclusiones podemos indicar si se ha cumplido o no el objetivo planteado. Por eso es necesario que el objetivo esté bien definido, porque si se acepta como objetivo válido en un proyecto, y éste se concluye como cumplido, el proyecto habrá sido ejecutado correctamente.

Ahora bien, ¿cómo determinamos que el objetivo se ha cumplido? pues intentando definirlo para que se pueda cumplir, es decir, intentando que sea:

\begin{itemize}
    \item \textbf{Acotado en el tiempo}, así es más fácil establecer un marco temporal para su realización y programar temporalmente las partes de las que se compone.
    \item \textbf{Medible}, para saber cómo de lejos estamos de llegar a un resultado aceptable.
    \item \textbf{Específico}, de manera que esté bien acotado y sea difícil embarcarse en tareas que no nos acerquen a su consecución.
    \item \textbf{Alcanzable}, porque si no lo es, por mucha intención y esfuerzo que le pongamos no se va a terminar.
    \item \textbf{Relevante}, porque si, en un \gls{pfg} para Ingeniería del Software, desarrollamos un producto mecánico para sexar pollos, pues por muy importante que sea, poco tiene que ver con lo que se ha estudiado durante todos estos años.
\end{itemize}

Y sí, para acordarnos de cuáles son estas características podemos usar el acrónimo \textit{AMEAR}.

\section{Motivación}

Qué factores han hecho al estudiante decantarse por trabajar en éste y no en otro tema.

Lo más indicado en este caso es apoyarse en datos de fuentes contrastables en lugar de en expresiones tipo ``ampliar mis conocimientos''. Información extraída de revistas especializadas (i.e. científicas), periódicos, organismos de estandarización, el \gls{ine}, etcétera se suele presuponer contrastada y fiable, y por tanto una buena base sobre la que partir.

\section{Justificación}

La justificación tiene como objetivo principal proporcionar una base sólida sobre el porqué de la realización del proyecto. En esta sección se debe explicar y argumentar las razones por las cuales se eligió el tema del proyecto, así como su importancia y relevancia. Algunos elementos clave que se pueden abordar en esta sección son:

\begin{enumerate}
    \item \textbf{Relevancia del tema}:: ¿Existe alguna necesidad o problema específico que tu proyecto pueda abordar?
    \item \textbf{Justificación teórica}: Mención sobre qué teorías, enfoques o modelos existentes en la literatura respalden la importancia de abordar este tema.
    \item \textbf{Brecha en el conocimiento}: ¿Qué aspectos no se han explorado lo suficiente o no han sido abordados en estudios previos? ¿Cómo puede el proyecto contribuir a cerrar esa brecha en el conocimiento?
    \item \textbf{Contribución práctica}: Aplicaciones prácticas de tu proyecto y cómo puede beneficiar a la comunidad académica, profesional o a la sociedad en general.
\end{enumerate}

La sección no tiene por qué ser demasiado extensa, ni tiene por qué incluir (o limitarse) a los puntos anteriores, pero debe ser lo suficientemente clara y convincente para que los lectores comprendan por qué el proyecto es relevante y necesario.

\section{Estructura de la memoria}

Cómo se organiza y estructura el proyecto en su totalidad. Esta sección presenta un resumen de los diferentes capítulos que conforman la memoria, así como una \textbf{muy} breve descripción de su contenido y propósito. Proporciona al lector una visión general de la estructura y el flujo del trabajo, permitiéndole comprender la secuencia lógica de cómo se desarrolla el trabajo o investigación.